\section{VHDL}
Jede VHDL Datei ist durch \textit{Library, Entity, Architecture} definiert. \textbf{Identifier} sind definiert durch:
\begin{itemize}[nosep]
	\item a..zA..Z0..9\_
	\item Start mit Buchstabe
	\item Case insensitive
	\item kein Schlüsselwort
\end{itemize}

\subsection{Library}
Eine Library ist eine Sammlung allgemeiner verwendeten, kompilierten und direkt ausführbaren Code-Blöcken. Dies erlaubt die Wiederverwendung von Code in verschiedenen Projekten.

\begin{lstlisting}
	library ieee; -- ieee standard und work library ohne qualifier verwendbar
	use ieee.std_logic_1164.all;
	use ieee.numeric_std.all;
\end{lstlisting}


\subsection{Entity}
Entity beschreibt eine Designeinheit als Black-Box. Im Fokus steht die Schnittstelle einer Entwurfseinheit nach aussen, sogennante \textbf{Ports}. Signale dürfen \underline{nicht} in der Port-Liste initialisiert werden.

\begin{lstlisting}
	entity <name> is 
	  port (
	    {<port_name> : <mode> <type>; }
	  );
	end <name>;
\end{lstlisting}
\noindent\textit{mode}
\begin{itemize}[nosep]
	\item \textbf{in}: Eingangsignal darf nur auf rechter Seite stehen
	\item \textbf{out}: Ausgangssignal darf nur auf linker Seite stehen
	\item \textbf{inout}: Bidirektionale Signale, wird mit \textit{std\_logic} verwendet
\end{itemize}

\subsection{Architecture}
Die architecture beschreibt den inneren Aufbau der Funktionalität, wobei zu einer Entity mehrere Architekturen gehören können. \textbf{Best Practice}: arch\_name ist oft einer der folgenden:
\begin{itemize}[nosep]
	\item \textbf{structural}: Ein und Ausgänge von Komponenten, die in einer Bibliotek abgelegt sind, mit lokalen Signalen verbinden. Netzliste aufbauen
	\item \textbf{behavioral}: Modellierung von Funktionalität
	\item \textbf{tb}: Test Beschreibung, modelliert nur Test-Setup
\end{itemize}

\begin{lstlisting}
	architecture <arch_name> of <entity_name> is
	-- Deklerationen...
	[type_decl]
	[subtype_decl]
	[constant_decl]
	[signal_decl]
	[component_decl]
	begin
	-- Anweisungen...
	end [<arch_name>];
\end{lstlisting}

\subsubsection{Component}
Komponenten (Entities) die innerhalb der Architektur instanziirt werden sollen, müssen deklariert sein.
\begin{lstlisting}
	component <name>
		port(...)
\end{lstlisting}

\subsubsection{Signal}
Interne Signale müssen in der Architecture definiert sein
\begin{lstlisting}
	signal <name> : <type>
\end{lstlisting}

\subsection{Prozesse}
Ein Prozess wird aktiviert, wenn ein Signal in der \textit{sensitivity list} geändert wird. Falls keine sensitivity Liste vorhanden ist, ist der Prozess IMMER aktiv (was nur in TB mit WAIT Statements benötigt wird). Signale werden am ende des Prozess geschrieben. 
\begin{lstlisting}
	[<label>:] process[<sensitivity list>]
	-- Deklerationen...
	[type_decl]
	[subtype_decl]
	[constant_decl]
	[signal_decl]
	[component_decl]
	begin
	-- Sequentielle Anweisungen...
	end process [<label>];
\end{lstlisting}

\subsection{Variable}
Variable sind nur innerhalb von Prozessen gültig. Sie sind nach Zuweisung sofort gültig und müssen nicht auf nächste Prozess Aktivierung warten. Diese sollten immer initialisiert werden, da ansonsten ein D-Latch erstellt wird.
\begin{lstlisting}
	variable <name>: <type> [:= <expression>]
\end{lstlisting}

\subsection{Type}
Aufzählungen werden von links nach rechts codiert zB mit Draycode. Angefangen mit $0$ bis $n$. Diese können in Vivado mit der entsprechende Option festgelegt werden (\textit{fsm\_extraction}).
\begin{lstlisting}
	type <type_name> is ({<identifier> ,?})
\end{lstlisting}

\subsection{Anweisungen}
\subsubsection{If}
If Anweisungen unbedingt mit else abschliessen, ansonsten wird ein Latch erstellt, welcher Hardware-Platz und zusätzliches Timing benötigt.

\subsubsection{case}
Case immer mit \textit{when others} abschliessen, siehe if.

\subsubsection{Zuweisung}
Ein \textit{signal} wird der Wert zugewiesen. Bit-Konstante sind mit ' und Vektoren mit \"\ zuzuweisen. Mit concatenate '\&' können signale verbunden werden. 
\begin{lstlisting}
	[<label>:]<signal> <= expression | "1010" | sig1 & "00" | (1=>s(1), 0 => s(0), others => '0') 
\end{lstlisting}
Selektive Zuweisung, sind typisch für kombinatorische Logik mit Verzweigung, müssen alle angegeben werden (mit others). Typisch für Multiplexer.
\begin{lstlisting}
	with <select_sig> select <ausgang_sig> <= 
		{ <eingang_sig> when "00", }
		when others;
\end{lstlisting}
\begin{lstlisting}
	<ausgang_sig> <= { <eingang_sig> when "00" else };
\end{lstlisting}
Diese zwei sind nicht identisch, weil erste parallel ausgeführt, während letztere sequentiell ausgewertet werden, was grossen Einfluss auf Laufzeit hat.
